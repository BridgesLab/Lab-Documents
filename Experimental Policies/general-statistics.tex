\documentclass[]{article}
\usepackage{lmodern}
\usepackage{amssymb,amsmath}
\usepackage{ifxetex,ifluatex}
\usepackage{fixltx2e} % provides \textsubscript
\ifnum 0\ifxetex 1\fi\ifluatex 1\fi=0 % if pdftex
  \usepackage[T1]{fontenc}
  \usepackage[utf8]{inputenc}
\else % if luatex or xelatex
  \ifxetex
    \usepackage{mathspec}
  \else
    \usepackage{fontspec}
  \fi
  \defaultfontfeatures{Ligatures=TeX,Scale=MatchLowercase}
\fi
% use upquote if available, for straight quotes in verbatim environments
\IfFileExists{upquote.sty}{\usepackage{upquote}}{}
% use microtype if available
\IfFileExists{microtype.sty}{%
\usepackage{microtype}
\UseMicrotypeSet[protrusion]{basicmath} % disable protrusion for tt fonts
}{}
\usepackage[margin=1in]{geometry}
\usepackage{hyperref}
\hypersetup{unicode=true,
            pdftitle={General Statistics},
            pdfauthor={Dave Bridges},
            pdfborder={0 0 0},
            breaklinks=true}
\urlstyle{same}  % don't use monospace font for urls
\usepackage{color}
\usepackage{fancyvrb}
\newcommand{\VerbBar}{|}
\newcommand{\VERB}{\Verb[commandchars=\\\{\}]}
\DefineVerbatimEnvironment{Highlighting}{Verbatim}{commandchars=\\\{\}}
% Add ',fontsize=\small' for more characters per line
\usepackage{framed}
\definecolor{shadecolor}{RGB}{248,248,248}
\newenvironment{Shaded}{\begin{snugshade}}{\end{snugshade}}
\newcommand{\KeywordTok}[1]{\textcolor[rgb]{0.13,0.29,0.53}{\textbf{{#1}}}}
\newcommand{\DataTypeTok}[1]{\textcolor[rgb]{0.13,0.29,0.53}{{#1}}}
\newcommand{\DecValTok}[1]{\textcolor[rgb]{0.00,0.00,0.81}{{#1}}}
\newcommand{\BaseNTok}[1]{\textcolor[rgb]{0.00,0.00,0.81}{{#1}}}
\newcommand{\FloatTok}[1]{\textcolor[rgb]{0.00,0.00,0.81}{{#1}}}
\newcommand{\ConstantTok}[1]{\textcolor[rgb]{0.00,0.00,0.00}{{#1}}}
\newcommand{\CharTok}[1]{\textcolor[rgb]{0.31,0.60,0.02}{{#1}}}
\newcommand{\SpecialCharTok}[1]{\textcolor[rgb]{0.00,0.00,0.00}{{#1}}}
\newcommand{\StringTok}[1]{\textcolor[rgb]{0.31,0.60,0.02}{{#1}}}
\newcommand{\VerbatimStringTok}[1]{\textcolor[rgb]{0.31,0.60,0.02}{{#1}}}
\newcommand{\SpecialStringTok}[1]{\textcolor[rgb]{0.31,0.60,0.02}{{#1}}}
\newcommand{\ImportTok}[1]{{#1}}
\newcommand{\CommentTok}[1]{\textcolor[rgb]{0.56,0.35,0.01}{\textit{{#1}}}}
\newcommand{\DocumentationTok}[1]{\textcolor[rgb]{0.56,0.35,0.01}{\textbf{\textit{{#1}}}}}
\newcommand{\AnnotationTok}[1]{\textcolor[rgb]{0.56,0.35,0.01}{\textbf{\textit{{#1}}}}}
\newcommand{\CommentVarTok}[1]{\textcolor[rgb]{0.56,0.35,0.01}{\textbf{\textit{{#1}}}}}
\newcommand{\OtherTok}[1]{\textcolor[rgb]{0.56,0.35,0.01}{{#1}}}
\newcommand{\FunctionTok}[1]{\textcolor[rgb]{0.00,0.00,0.00}{{#1}}}
\newcommand{\VariableTok}[1]{\textcolor[rgb]{0.00,0.00,0.00}{{#1}}}
\newcommand{\ControlFlowTok}[1]{\textcolor[rgb]{0.13,0.29,0.53}{\textbf{{#1}}}}
\newcommand{\OperatorTok}[1]{\textcolor[rgb]{0.81,0.36,0.00}{\textbf{{#1}}}}
\newcommand{\BuiltInTok}[1]{{#1}}
\newcommand{\ExtensionTok}[1]{{#1}}
\newcommand{\PreprocessorTok}[1]{\textcolor[rgb]{0.56,0.35,0.01}{\textit{{#1}}}}
\newcommand{\AttributeTok}[1]{\textcolor[rgb]{0.77,0.63,0.00}{{#1}}}
\newcommand{\RegionMarkerTok}[1]{{#1}}
\newcommand{\InformationTok}[1]{\textcolor[rgb]{0.56,0.35,0.01}{\textbf{\textit{{#1}}}}}
\newcommand{\WarningTok}[1]{\textcolor[rgb]{0.56,0.35,0.01}{\textbf{\textit{{#1}}}}}
\newcommand{\AlertTok}[1]{\textcolor[rgb]{0.94,0.16,0.16}{{#1}}}
\newcommand{\ErrorTok}[1]{\textcolor[rgb]{0.64,0.00,0.00}{\textbf{{#1}}}}
\newcommand{\NormalTok}[1]{{#1}}
\usepackage{longtable,booktabs}
\usepackage{graphicx,grffile}
\makeatletter
\def\maxwidth{\ifdim\Gin@nat@width>\linewidth\linewidth\else\Gin@nat@width\fi}
\def\maxheight{\ifdim\Gin@nat@height>\textheight\textheight\else\Gin@nat@height\fi}
\makeatother
% Scale images if necessary, so that they will not overflow the page
% margins by default, and it is still possible to overwrite the defaults
% using explicit options in \includegraphics[width, height, ...]{}
\setkeys{Gin}{width=\maxwidth,height=\maxheight,keepaspectratio}
\IfFileExists{parskip.sty}{%
\usepackage{parskip}
}{% else
\setlength{\parindent}{0pt}
\setlength{\parskip}{6pt plus 2pt minus 1pt}
}
\setlength{\emergencystretch}{3em}  % prevent overfull lines
\providecommand{\tightlist}{%
  \setlength{\itemsep}{0pt}\setlength{\parskip}{0pt}}
\setcounter{secnumdepth}{5}
% Redefines (sub)paragraphs to behave more like sections
\ifx\paragraph\undefined\else
\let\oldparagraph\paragraph
\renewcommand{\paragraph}[1]{\oldparagraph{#1}\mbox{}}
\fi
\ifx\subparagraph\undefined\else
\let\oldsubparagraph\subparagraph
\renewcommand{\subparagraph}[1]{\oldsubparagraph{#1}\mbox{}}
\fi

%%% Use protect on footnotes to avoid problems with footnotes in titles
\let\rmarkdownfootnote\footnote%
\def\footnote{\protect\rmarkdownfootnote}

%%% Change title format to be more compact
\usepackage{titling}

% Create subtitle command for use in maketitle
\newcommand{\subtitle}[1]{
  \posttitle{
    \begin{center}\large#1\end{center}
    }
}

\setlength{\droptitle}{-2em}
  \title{General Statistics}
  \pretitle{\vspace{\droptitle}\centering\huge}
  \posttitle{\par}
  \author{Dave Bridges}
  \preauthor{\centering\large\emph}
  \postauthor{\par}
  \predate{\centering\large\emph}
  \postdate{\par}
  \date{May 9, 2014}


\begin{document}
\maketitle

{
\setcounter{tocdepth}{2}
\tableofcontents
}
\section{General Statistical Methods}\label{general-statistical-methods}

There are several important concepts that we will adhere to in our
group. These involve design considerations, execution considerations and
analysis concerns.

\subsection{Experimental Design}\label{experimental-design}

Where possible, prior to performing an experiment or study perform a
power analysis. This is mainly to determine the appropriate sample
sizes. To do this, you need to know a few of things:

\begin{itemize}
\tightlist
\item
  Either the sample size or the difference. The difference is provided
  in standard deviations. This means that you need to know the standard
  deviation of your measurement in question. It is a good idea to keep a
  log of these for your data, so that you can approximate what this is.
  If you hope to detect a correlation you will need to know the expected
  correlation coefficient.
\item
  The desired false positive rate (normally 0.05). This is the rate at
  which you find a difference where there is none. This is also known as
  the type I error rate.
\item
  The desired power (normally 0.8). This indicates that 80\% of the time
  you will detect the effect if there is one. This is also known as 1
  minus the false negative rate or 1 minus the Type II error rate.
\end{itemize}

We use the R package \textbf{pwr} to do a power analysis (Champely,
2017). Here is an example:

\subsubsection{Pairwise Comparasons}\label{pairwise-comparasons}

\begin{Shaded}
\begin{Highlighting}[]
\KeywordTok{require}\NormalTok{(pwr)}
\NormalTok{false.negative.rate <-}\StringTok{ }\FloatTok{0.05}
\NormalTok{statistical.power <-}\StringTok{ }\FloatTok{0.8}
\NormalTok{sd <-}\StringTok{ }\FloatTok{3.5} \CommentTok{#this is calculated from known measurements}
\NormalTok{difference <-}\StringTok{ }\DecValTok{3}  \CommentTok{#you hope to detect a difference }
\KeywordTok{pwr.t.test}\NormalTok{(}\DataTypeTok{d =} \NormalTok{difference, }\DataTypeTok{sig.level =} \NormalTok{false.negative.rate, }\DataTypeTok{power=}\NormalTok{statistical.power)}
\end{Highlighting}
\end{Shaded}

\begin{verbatim}
## 
##      Two-sample t test power calculation 
## 
##               n = 3.07
##               d = 3
##       sig.level = 0.05
##           power = 0.8
##     alternative = two.sided
## 
## NOTE: n is number in *each* group
\end{verbatim}

This tells us that in order to see a difference of at least 3, with at
standard devation of 3.5 we need at least \textbf{3} observations in
each group.

\subsubsection{Correlations}\label{correlations}

The following is an example for detecting a correlation.

\begin{Shaded}
\begin{Highlighting}[]
\KeywordTok{require}\NormalTok{(pwr)}
\NormalTok{false.negative.rate <-}\StringTok{ }\FloatTok{0.05}
\NormalTok{statistical.power <-}\StringTok{ }\FloatTok{0.8}
\NormalTok{correlation.coefficient <-}\StringTok{ }\FloatTok{0.6} \CommentTok{#note that this is the r, to get the R2 value you will have to square this result.}
\KeywordTok{pwr.r.test}\NormalTok{(}\DataTypeTok{r =} \NormalTok{correlation.coefficient, }\DataTypeTok{sig.level =} \NormalTok{false.negative.rate, }\DataTypeTok{power=}\NormalTok{statistical.power)}
\end{Highlighting}
\end{Shaded}

\begin{verbatim}
## 
##      approximate correlation power calculation (arctangh transformation) 
## 
##               n = 18.6
##               r = 0.6
##       sig.level = 0.05
##           power = 0.8
##     alternative = two.sided
\end{verbatim}

This tells us that in order to detect a correlation coefficient of at
least 0.6 (or an R\^{}2 of 0.36) you need more than \textbf{18}
observations.

\section{Pairwise Testing}\label{pairwise-testing}

If you have two groups (and two groups only) that you want to know if
they are different, you will normally want to do a pairwise test. This
is \textbf{not} the case if you have paired data (before and after for
example). The most common of these is something called a Student's
\emph{t}-test, but this test has two key assumptions:

\begin{itemize}
\tightlist
\item
  The data are normally distributed
\item
  The two groups have equal variance
\end{itemize}

\subsection{Testing the Assumptions}\label{testing-the-assumptions}

Best practice is to first test for normality, and if that test passes,
to then test for equal variance

\subsubsection{Testing Normality}\label{testing-normality}

To test normality, we use a Shapiro-Wilk test (details on
\href{https://en.wikipedia.org/wiki/Shapiro\%E2\%80\%93Wilk_test}{Wikipedia}
on each of your two groups). Below is an example where there are two
groups:

\begin{Shaded}
\begin{Highlighting}[]
\CommentTok{#create seed for reproducibility}
\KeywordTok{set.seed}\NormalTok{(}\DecValTok{1265}\NormalTok{)}
\NormalTok{test.data <-}\StringTok{ }\KeywordTok{tibble}\NormalTok{(}\DataTypeTok{Treatment=}\KeywordTok{c}\NormalTok{(}\KeywordTok{rep}\NormalTok{(}\StringTok{"Experiment"}\NormalTok{,}\DecValTok{6}\NormalTok{), }\KeywordTok{rep}\NormalTok{(}\StringTok{"Control"}\NormalTok{,}\DecValTok{6}\NormalTok{)),}
           \DataTypeTok{Result =} \KeywordTok{rnorm}\NormalTok{(}\DataTypeTok{n=}\DecValTok{12}\NormalTok{, }\DataTypeTok{mean=}\DecValTok{10}\NormalTok{, }\DataTypeTok{sd=}\DecValTok{3}\NormalTok{))}
\CommentTok{#test.data$Treatment <- as.factor(test.data$Treatment)}
\KeywordTok{kable}\NormalTok{(test.data, }\DataTypeTok{caption=}\StringTok{"The test data used in the following examples"}\NormalTok{)}
\end{Highlighting}
\end{Shaded}

\begin{longtable}[]{@{}lr@{}}
\caption{The test data used in the following examples}\tabularnewline
\toprule
Treatment & Result\tabularnewline
\midrule
\endfirsthead
\toprule
Treatment & Result\tabularnewline
\midrule
\endhead
Experiment & 11.26\tabularnewline
Experiment & 8.33\tabularnewline
Experiment & 9.94\tabularnewline
Experiment & 11.83\tabularnewline
Experiment & 6.56\tabularnewline
Experiment & 11.41\tabularnewline
Control & 8.89\tabularnewline
Control & 11.59\tabularnewline
Control & 9.39\tabularnewline
Control & 8.74\tabularnewline
Control & 6.31\tabularnewline
Control & 7.82\tabularnewline
\bottomrule
\end{longtable}

Each of the two groups, in this case \textbf{Test} and \textbf{Control}
must have Shapiro-Wilk tests done separately. Some sample code for this
is below (requires dplyr to be loaded):

\begin{Shaded}
\begin{Highlighting}[]
\CommentTok{#filter only for the control data}
\NormalTok{control.data <-}\StringTok{ }\KeywordTok{filter}\NormalTok{(test.data, Treatment==}\StringTok{"Control"}\NormalTok{)}
\CommentTok{#The broom package makes the results of the test appear in a table, with the tidy command}
\KeywordTok{library}\NormalTok{(broom)}

\CommentTok{#run the Shapiro-Wilk test on the values}
\KeywordTok{shapiro.test}\NormalTok{(control.data$Result) %>%}\StringTok{ }\NormalTok{tidy %>%}\StringTok{ }\NormalTok{kable}
\end{Highlighting}
\end{Shaded}

\begin{longtable}[]{@{}rrl@{}}
\toprule
statistic & p.value & method\tabularnewline
\midrule
\endhead
0.968 & 0.88 & Shapiro-Wilk normality test\tabularnewline
\bottomrule
\end{longtable}

\begin{Shaded}
\begin{Highlighting}[]
\NormalTok{experiment.data <-}\StringTok{ }\KeywordTok{filter}\NormalTok{(test.data, Treatment==}\StringTok{"Experiment"}\NormalTok{)}
\KeywordTok{shapiro.test}\NormalTok{(test.data$Result) %>%}\StringTok{ }\NormalTok{tidy %>%}\StringTok{ }\NormalTok{kable}
\end{Highlighting}
\end{Shaded}

\begin{longtable}[]{@{}rrl@{}}
\toprule
statistic & p.value & method\tabularnewline
\midrule
\endhead
0.93 & 0.377 & Shapiro-Wilk normality test\tabularnewline
\bottomrule
\end{longtable}

Based on these results, since both p-values are \textgreater{}0.05 we do
not reject the presumption of normality and can go on. If one or more of
the p-values were less than 0.05 we would then use a Mann-Whitney test
(also known as a Wilcoxon rank sum test) will be done, see below for
more details.

\subsubsection{Testing for Equal
Variance}\label{testing-for-equal-variance}

We generally use the
\href{https://cran.r-project.org/web/packages/car/index.html}{car}
package which contains code for
\href{https://en.wikipedia.org/wiki/Levene\%27s_test}{Levene's Test} to
see if two groups can be assumed to have equal variance:

\begin{Shaded}
\begin{Highlighting}[]
\CommentTok{#load the car package}
\KeywordTok{library}\NormalTok{(car)}

\CommentTok{#runs the test, grouping by the Treatment variable}
\KeywordTok{leveneTest}\NormalTok{(Result ~}\StringTok{ }\NormalTok{Treatment, }\DataTypeTok{data=}\NormalTok{test.data) %>%}\StringTok{ }\NormalTok{tidy %>%}\StringTok{ }\NormalTok{kable}
\end{Highlighting}
\end{Shaded}

\begin{longtable}[]{@{}lrrr@{}}
\toprule
term & df & statistic & p.value\tabularnewline
\midrule
\endhead
group & 1 & 0.368 & 0.558\tabularnewline
& 10 & NA & NA\tabularnewline
\bottomrule
\end{longtable}

\subsection{Performing the Appropriate Pairwise
Test}\label{performing-the-appropriate-pairwise-test}

The logic to follow is:

\begin{itemize}
\tightlist
\item
  If the Shapiro-Wilk test passes, do Levene's test. If it fails for
  either group, move on to a \textbf{Wilcoxon Rank Sum Test}.
\item
  If Levene's test \emph{passes}, do a Student's \emph{t} Test, which
  assumes equal variance.
\item
  If Levene's test \emph{fails}, do a Welch's \emph{t} Test, which does
  not assume equal variance.
\end{itemize}

\subsubsection{\texorpdfstring{Student's \emph{t}
Test}{Student's t Test}}\label{students-t-test}

\begin{Shaded}
\begin{Highlighting}[]
\CommentTok{#The default for t.test in R is Welch's, so you need to set the var.equal variable to be TRUE}
\KeywordTok{t.test}\NormalTok{(Result~Treatment,}\DataTypeTok{data=}\NormalTok{test.data, }\DataTypeTok{var.equal=}\NormalTok{T) %>%}\StringTok{ }\NormalTok{tidy %>%}\StringTok{ }\NormalTok{kable}
\end{Highlighting}
\end{Shaded}

\begin{longtable}[]{@{}rrrrrrrll@{}}
\toprule
estimate1 & estimate2 & statistic & p.value & parameter & conf.low &
conf.high & method & alternative\tabularnewline
\midrule
\endhead
8.79 & 9.89 & -0.992 & 0.345 & 10 & -3.56 & 1.37 & Two Sample t-test &
two.sided\tabularnewline
\bottomrule
\end{longtable}

\subsubsection{\texorpdfstring{Welch's \emph{t}
Test}{Welch's t Test}}\label{welchs-t-test}

\begin{Shaded}
\begin{Highlighting}[]
\CommentTok{#The default for t.test in R is Welch's, so you need to set the var.equal variable to be FALSE, or leave the default}
\KeywordTok{t.test}\NormalTok{(Result~Treatment,}\DataTypeTok{data=}\NormalTok{test.data, }\DataTypeTok{var.equal=}\NormalTok{F) %>%}\StringTok{ }\NormalTok{tidy %>%}\StringTok{ }\NormalTok{kable}
\end{Highlighting}
\end{Shaded}

\begin{longtable}[]{@{}rrrrrrrrll@{}}
\toprule
estimate & estimate1 & estimate2 & statistic & p.value & parameter &
conf.low & conf.high & method & alternative\tabularnewline
\midrule
\endhead
-1.1 & 8.79 & 9.89 & -0.992 & 0.345 & 9.72 & -3.57 & 1.38 & Welch Two
Sample t-test & two.sided\tabularnewline
\bottomrule
\end{longtable}

\subsubsection{Wilcoxon Rank Sum Test}\label{wilcoxon-rank-sum-test}

\begin{Shaded}
\begin{Highlighting}[]
\CommentTok{# no need to specify anything about variance}
\KeywordTok{wilcox.test}\NormalTok{(Result~Treatment,}\DataTypeTok{data=}\NormalTok{test.data) %>%}\StringTok{ }\NormalTok{tidy %>%}\StringTok{ }\NormalTok{kable}
\end{Highlighting}
\end{Shaded}

\begin{longtable}[]{@{}rrll@{}}
\toprule
statistic & p.value & method & alternative\tabularnewline
\midrule
\endhead
12 & 0.394 & Wilcoxon rank sum test & two.sided\tabularnewline
\bottomrule
\end{longtable}

\section{Corrections for Multiple
Observations}\label{corrections-for-multiple-observations}

The best illustration I have seen for the need for multiple observation
corrections is this cartoon from XKCD (see \url{http://xkcd.com/882/}):

\begin{figure}[htbp]
\centering
\includegraphics{http://imgs.xkcd.com/comics/significant.png}
\caption{Significance by XKCD. Image is from
\url{http://imgs.xkcd.com/comics/significant.png}}
\end{figure}

Any conceptually coherent set of observations must therefore be
corrected for multiple observations. In most cases, we will use the
method of Benjamini and Hochberg since our p-values are not entirely
independent. Some sample code for this is here:

\begin{Shaded}
\begin{Highlighting}[]
\NormalTok{p.values <-}\StringTok{ }\KeywordTok{c}\NormalTok{(}\FloatTok{0.023}\NormalTok{, }\FloatTok{0.043}\NormalTok{, }\FloatTok{0.056}\NormalTok{, }\FloatTok{0.421}\NormalTok{, }\FloatTok{0.012}\NormalTok{)}
\KeywordTok{data.frame}\NormalTok{(}\DataTypeTok{unadjusted =} \NormalTok{p.values, }\DataTypeTok{adjusted=}\KeywordTok{p.adjust}\NormalTok{(p.values, }\DataTypeTok{method=}\StringTok{"BH"}\NormalTok{))}
\end{Highlighting}
\end{Shaded}

\begin{verbatim}
##   unadjusted adjusted
## 1      0.023   0.0575
## 2      0.043   0.0700
## 3      0.056   0.0700
## 4      0.421   0.4210
## 5      0.012   0.0575
\end{verbatim}

\section{Session Information}\label{session-information}

\begin{Shaded}
\begin{Highlighting}[]
\KeywordTok{sessionInfo}\NormalTok{()}
\end{Highlighting}
\end{Shaded}

\begin{verbatim}
## R version 3.4.2 (2017-09-28)
## Platform: x86_64-apple-darwin15.6.0 (64-bit)
## Running under: macOS High Sierra 10.13.1
## 
## Matrix products: default
## BLAS: /Library/Frameworks/R.framework/Versions/3.4/Resources/lib/libRblas.0.dylib
## LAPACK: /Library/Frameworks/R.framework/Versions/3.4/Resources/lib/libRlapack.dylib
## 
## locale:
## [1] en_US.UTF-8/en_US.UTF-8/en_US.UTF-8/C/en_US.UTF-8/en_US.UTF-8
## 
## attached base packages:
## [1] stats     graphics  grDevices utils     datasets  methods   base     
## 
## other attached packages:
## [1] car_2.1-6           broom_0.4.3         bindrcpp_0.2       
## [4] pwr_1.2-1           knitcitations_1.0.9 dplyr_0.7.4        
## [7] tidyr_0.7.2         knitr_1.17         
## 
## loaded via a namespace (and not attached):
##  [1] Rcpp_0.12.14       nloptr_1.0.4       compiler_3.4.2    
##  [4] plyr_1.8.4         highr_0.6          bindr_0.1         
##  [7] tools_3.4.2        lme4_1.1-14        digest_0.6.12     
## [10] jsonlite_1.5       lubridate_1.7.1    evaluate_0.10.1   
## [13] tibble_1.3.4       nlme_3.1-131       lattice_0.20-35   
## [16] mgcv_1.8-22        pkgconfig_2.0.1    rlang_0.1.4       
## [19] Matrix_1.2-12      psych_1.7.8        bibtex_0.4.2      
## [22] yaml_2.1.15        parallel_3.4.2     SparseM_1.77      
## [25] RefManageR_0.14.20 stringr_1.2.0      httr_1.3.1        
## [28] xml2_1.1.1         MatrixModels_0.4-1 nnet_7.3-12       
## [31] rprojroot_1.2      grid_3.4.2         glue_1.2.0        
## [34] R6_2.2.2           foreign_0.8-69     rmarkdown_1.8     
## [37] minqa_1.2.4        reshape2_1.4.2     purrr_0.2.4       
## [40] magrittr_1.5       splines_3.4.2      MASS_7.3-47       
## [43] backports_1.1.1    htmltools_0.3.6    pbkrtest_0.4-7    
## [46] assertthat_0.2.0   mnormt_1.5-5       quantreg_5.34     
## [49] stringi_1.1.6
\end{verbatim}

\section{References}\label{references}

\protect\hyperlink{cite-pwr}{{[}1{]}} S. Champely. \emph{pwr: Basic
Functions for Power Analysis}. R package version 1.2-1. 2017. URL:
\url{https://CRAN.R-project.org/package=pwr}.


\end{document}
